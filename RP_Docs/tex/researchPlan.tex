% Options for packages loaded elsewhere
\PassOptionsToPackage{unicode}{hyperref}
\PassOptionsToPackage{hyphens}{url}
%
\documentclass[
]{article}
\usepackage{amsmath,amssymb}
\usepackage{lmodern}
\usepackage{iftex}
\ifPDFTeX
  \usepackage[T1]{fontenc}
  \usepackage[utf8]{inputenc}
  \usepackage{textcomp} % provide euro and other symbols
\else % if luatex or xetex
  \usepackage{unicode-math}
  \defaultfontfeatures{Scale=MatchLowercase}
  \defaultfontfeatures[\rmfamily]{Ligatures=TeX,Scale=1}
\fi
% Use upquote if available, for straight quotes in verbatim environments
\IfFileExists{upquote.sty}{\usepackage{upquote}}{}
\IfFileExists{microtype.sty}{% use microtype if available
  \usepackage[]{microtype}
  \UseMicrotypeSet[protrusion]{basicmath} % disable protrusion for tt fonts
}{}
\makeatletter
\@ifundefined{KOMAClassName}{% if non-KOMA class
  \IfFileExists{parskip.sty}{%
    \usepackage{parskip}
  }{% else
    \setlength{\parindent}{0pt}
    \setlength{\parskip}{6pt plus 2pt minus 1pt}}
}{% if KOMA class
  \KOMAoptions{parskip=half}}
\makeatother
\usepackage{xcolor}
\usepackage{longtable,booktabs,array}
\usepackage{multirow}
\usepackage{calc} % for calculating minipage widths
% Correct order of tables after \paragraph or \subparagraph
\usepackage{etoolbox}
\makeatletter
\patchcmd\longtable{\par}{\if@noskipsec\mbox{}\fi\par}{}{}
\makeatother
% Allow footnotes in longtable head/foot
\IfFileExists{footnotehyper.sty}{\usepackage{footnotehyper}}{\usepackage{footnote}}
\makesavenoteenv{longtable}
\setlength{\emergencystretch}{3em} % prevent overfull lines
\providecommand{\tightlist}{%
  \setlength{\itemsep}{0pt}\setlength{\parskip}{0pt}}
\setcounter{secnumdepth}{-\maxdimen} % remove section numbering
\ifLuaTeX
  \usepackage{selnolig}  % disable illegal ligatures
\fi
\IfFileExists{bookmark.sty}{\usepackage{bookmark}}{\usepackage{hyperref}}
\IfFileExists{xurl.sty}{\usepackage{xurl}}{} % add URL line breaks if available
\urlstyle{same} % disable monospaced font for URLs
\hypersetup{
  hidelinks,
  pdfcreator={LaTeX via pandoc}}

\author{}
\date{}

\begin{document}

Research Plan for CSE3000 Research Project

Vladimir Sachkov

November 17, 2024

\textbf{1 Background of the research}

Current visual AI for hand landmark prediction is trained on normal
color images (RGB). We, however, wish to make predictions in the
infrared domain. One way to handle such a domain shift is by "domain
adaptation," and here we will explore its suitability. Irene Schemkes
has also conducted work on this topic in her master's thesis, focusing
on diagnosing leprosy with hand landmark tracking Schemkes, 2024.

\textbf{2 Background of the research}

Early diagnosis of leprosy, is essential to prevent transmission and
permanent nerve and tissue damage. Accurate temperature measurement in
regions of interest (ROIs) on the hands is a promising diagnostic
method, as current techniques often detect the disease only in later
stages when physical symptoms become apparent.

Manual temperature analysis involves time-consuming annotation of ROIs
and subsequent measurements, leading to potential human error and
inconsistencies.

Automated temperature measurement using artificial intelligence (AI)
addresses these challenges by enhancing the speed and accuracy of ROI
(Region of interest) detection and temperature analysis. AI-driven
solutions ensure consistent results across large datasets, minimize
human error, and employ real-time monitoring. Irene Schemkes conducted
similar work in her master's thesis, developing a semi-automatic
temperature analysis method utilizing real-time hand landmark tracking
in infrared videos Schemkes, 2024. Irene used Google's MediaPipe Hand
Landmarker detector to track hand landmarks in infrared videos AI, 2024,
but the problem is that the model was trained on RGB images, and not
infrared. Although she has used some adaptation techniques, such as
adjusting sensitivity of landmark detection thresholds, or applying
filters and transformations that highlight temperature variations, I
belieb there is still a big room for improvement in domain adaptation
part of the problem, that I am going to address and investigate in my
research.

\textbf{3 Research Question}

\textbf{Main Question:} How effective is domain adaptation in enhancing
the performance of visual hand landmark prediction AI when transitioning
from RGB to infrared (IR) imaging

1

domains?

Addressing the domain shift from RGB to IR is crucial for leveraging
AI's capabilities in medical diagnostics in general. This research aims
to evaluate its suitability and effective-ness in this context. I will
use Irene's work domain adaptation techniques as a baseline to compare
the performance of the proposed methods. This project's peer Zofia is
going to be answering a similarally stated research question, but
instead of domain adaptation tecniques she is going to be using style
transfer to convert IR videos to RGB, so in the later stages of the
project I am planning to conduct a joint investigation on comparing the
two methods, their resulting accuracy, resource usage, scalability, and
speed. It's also important to note, that main objective of the problem
is to optimize accuracy of predicted regions of interest on hands from
the entire video, and not just single frames. For example in Irene's
work, for example, Coordinate Filtering Techniques are used in the
context of the entire video. But in my work I will mostly focus on
single frames, and try to improve the accuracy of landmark detection.
Another problem is the absence of the ground-truth landmark points in
existing dataset. The established area of work totally fits the
time-frame for the research project, since it utilizes established
domain adaptation techniques and uses existing tools like MediaPipe and
machine learning libraries like TensorFlow and PyTorch.

\textbf{Sub-Questions:}

\begin{quote}
1. \textbf{What are the limitations of Irene's domain adaptation
techniques under current computational power constraints?}

• \emph{Objective Criteria:} Analyze performance bottlenecks, resource
usage, and scal- ability issues in her approach.

2. \textbf{What key identifiers will describe model's ability to perform
in our given scenario in Nepal?}

• \emph{Objective Criteria:} Discovered set of key identifiers should be
able to properly compare multiple pipelines for IR to RGB conversition.
For example the style transfer solution that will be produced by my peer
in a later stages. Key identifiers might include: resources and time
needed to retran, or fine-tune a model, inference resource usage and
time, accuracy, generalizability, and scalability.

3. \textbf{How can hand landmark detection accuracy be effectively
evaluated us-ing improved domain adaptation methods, considering
per-frame and temporal metrics?}

• \emph{Objective Criteria:} Compare per-frame and temporal accuracy
metrics, and identify any additional metrics required beyond those used
in Irene's work.

• Cross-validation using expert-annotated subset of frames

• Temporal consistency metrics across video sequences

• relative positions and distances between landmarks

• Stability analysis across different thermal conditions and hand
positions

4. \textbf{Will the low-cost methods (Feature Alignment Using Maximum
Mean Discrepance, Fine-Tuning with Pseudo-Labeling, or Adaptive Batch
Nor-}
\end{quote}

2

\begin{quote}
\textbf{malization) be able to improve the performance of hand landmark
pre-diction in infrared videos compared to Irene's baseline?}

• \emph{Objective Criteria:} The low-cost methods implemnted only using
compute re-sources available online will provide better results than
Irene's work, given only infrared videos dataset.

5. \textbf{Will the medium-cost method (Domain-Adversarial Neural
Networks domain adaptation method, Correlation Alignment, or
Self-Supervised Learning with Pretext Tasks) produce better and faster
inference results than Irene's baseline or cheaper methods?}

• \emph{Objective Criteria:} Implement and assess specific domain
adaptation tech- niques to enhance model performance on infrared (IR)
data using comparative analysis with metrics such as accuracy,
precision, recall, and F1-score.

6. \textbf{Will the domain adaptation method result in a model that
maintains rea-sonable inference time for real-time hand landmark
prediction in infrared videos?}

• \emph{Objective Criteria:} The adapted model's inference time and
resource utiliza-tion should not exceed predefined thresholds that
ensure real-time processing capabilities. Irene's method averaged 42
minutes per video processing time.
\end{quote}

\textbf{4 Method}

To tackle the research aims, I'll break down the method into smaller,
doable tasks, tackling each sub-question one by one in the order they're
listed.

**1. Setting Up the Development Environment** First, I need to figure
out the best development environment for this project. I'll evaluate
different machine learning libraries to ensure they have all the
necessary data processing methods and advanced mathematical models.
Libraries like TensorFlow, PyTorch, and scikit-learn will be considered
based on their capabilities and compatibility with our goals.

**2. Organizing and Cleaning Data** Next, I'll organize and clean the
existing infrared video data. This involves manually labeling the points
of interest on the hands to create accurate datasets. Additionally, I'll
add metadata to videos and frames, such as relative hand positions and
other relevant information, to enrich the dataset.

**3. Feature Extraction with Unsupervised Learning** I'll run some
unsupervised ma-chine learning methods to identify unique features in
the videos. Techniques like clustering and dimensionality reduction will
help in uncovering patterns and features that are not immediately
apparent.

**4. Developing a Custom Evaluation Framework** Implementing a custom
evaluation framework is crucial for testing solutions. This framework
will assess accuracy, perform cross-validation, and more, addressing the
second sub-question. To ensure robustness, I might include testing on
custom videos captured with our own camera to check if the models
generalize well.

**5. Implementing Irene's Methods** I'll start by implementing Irene's
domain adapta-tion methods, including Google's MediaPipe Hand
Landmarker. Running these methods

3

through my evaluation pipeline will allow me to compare the performance
metrics with those reported in Irene's paper.

**6. Identifying and Addressing Limitations** After evaluating Irene's
methods, I'll pin-point their limitations. Based on these insights, I'll
begin implementing low-cost domain adaptation techniques specifically
designed to overcome these issues. If certain challenges require more
advanced solutions, such as deep learning or fine-tuning, I'll seek
access to a computation cluster from Delft.

**7. Experimenting with Various Domain Adaptation Techniques** I'll
explore a range of domain adaptation methods, from cheap options like
Feature Alignment using Maximum Mean Discrepancy and Fine-Tuning with
Pseudo-Labeling to medium-cost methods such as Domain-Adversarial Neural
Networks and Self-Supervised Learning with Pretext Tasks. Each method
will be evaluated based on accuracy, hardware resource usage, and
processing time.

**8. Collaboration and Optimization** Throughout the project, I'll
collaborate with peers experienced in machine learning to troubleshoot
and optimize the models. Using version control tools like Git will help
manage the codebase efficiently and track changes.

**9. Validation and Documentation** Each task's completion will be
validated by com-paring the performance metrics against Irene's baseline
using scikit-learn. Finally, I'll document the entire process, ensuring
that each step meets the predefined objectives and preparing the results
for further analysis. Once having propotypes of my domain adap-tation
methods, I will be ready to compare their performance with Zofia's style
transfer technique.

\textbf{5 Planning of the Research Project}

\textbf{References}

AI, G. (2024). Mediapipe: Hand landmarker {[}Accessed: 2024-05-21{]}.

Schemkes, I. (2024). \emph{Diagnosing leprosy with hand landmark
tracking} {[}Master's thesis, Delft University of Technology{]}
{[}Unpublished master's thesis{]}.

4

\begin{longtable}{p{0.2\textwidth} p{0.5\textwidth} p{0.3\textwidth}}
\toprule
\textbf{Date} & \textbf{Activity} & \textbf{Type} \\
\midrule
\endhead

11 Nov 2024 - 17 Nov 2024 & Conduct systematic literature review and develop evaluation framework & Personal Work \\

11 Nov 2024 & Lecture - Information Literacy Assignment & Lecture \\

11 Nov 2024 & Lecture - Kick-Off & Lecture \\

18 Nov 2024 - 24 Nov 2024 & Finalize systematic literature review and complete evaluation framework & Personal Work \\

12 Nov 2024 & Student Deadline - First Week Plan & Student Deadline \\

12 Nov 2024 & Supervisor - Meeting & Meeting \\

15 Nov 2024 & Lecture - ACS Assignment 1 & Lecture \\

21 Nov 2024 & Research plan meeting & Meeting \\

22 Nov 2024 & Responsible Professor - Research plan presentation & Meeting \\

25 Nov 2024 & Lecture - Session Responsible Research & Lecture \\

25 Nov 2024 & Lecture - Session ACS (paper) & Lecture \\

25 Nov 2024 - 1 Dec 2024 & Develop code for testing and comparing models and prepare clean annotated data & Personal Work \\

29 Nov 2024 & Student Deadline - Set Date for Final Presentation & Student Deadline \\

29 Nov 2024 & Supervisor - Meeting & Meeting \\

29 Nov 2024 & Responsible Professor - Meeting & Meeting \\

29 Nov 2024 & Examiner - Meeting & Meeting \\

2 Dec 2024 - 8 Dec 2024 & Implement and evaluate cheaper methods (and Irene's domain adaptation techniques), and assess feasibility of medium resource demand methods & Personal Work \\

2 Dec 2024 & Lecture - Session Responsible Research & Lecture \\

6 Dec 2024 & Lecture - ACS Assignments 2a and 2b & Lecture \\

9 Dec 2024 & Lecture - Session ACS (poster) & Lecture \\

10 Dec 2024 & Midterm meeting & Meeting \\

11 Dec 2024 & Student Deadline - Midterm Presentations & Student Deadline \\

13 Dec 2024 & Lecture - ACS Assignment 3 & Lecture \\

13 Dec 2024 - 15 Dec 2024 & Write introduction and methodology sections; prepare rough outline for midterm presentation & Personal Work \\

16 Dec 2024 & Lecture - Session ACS (paper) & Lecture \\

16 Dec 2024 - 22 Dec 2024 & Make custom model functional before Midterm Presentation & Personal Work \\

23 Dec 2024 & Lecture - Coaching ACS & Lecture \\

24 Dec 2024 & Student Deadline - Paper Draft v1 & Student Deadline \\

25 Dec 2024 & Conduct experiments and fix bugs in time for Paper Draft v1 & Personal Work \\

26 Dec 2024 & Student Deadline - Peer Review Draft v1 & Student Deadline \\

30 Dec 2024 & Lecture - Coaching ACS & Lecture \\

30 Dec 2024 - 5 Jan 2025 & Rewrite paper using peer review feedback; amend potential shortcomings & Personal Work \\

1 Jan 2025 & Student Deadline - Paper Draft v2 & Student Deadline \\

6 Jan 2025 & Lecture - Coaching ACS & Lecture \\

6 Jan 2025 - 12 Jan 2025 & Ensure proper styling, citation, and minor details for Paper Draft v2; prepare rough draft of the poster & Personal Work \\

7 Jan 2025 & Responsible Professor - Feedback v2 & Meeting \\

8 Jan 2025 & Student Deadline - Submit Final Paper & Student Deadline \\

13 Jan 2025 & Lecture - Session ACS (poster) & Lecture \\

13 Jan 2025 - 17 Jan 2025 & Prepare rough draft of the poster and prepare for the presentation & Personal Work \\

13 Jan 2025 & Student Deadline - Submit Poster & Student Deadline \\

17 Jan 2025 & Student Deadline - Poster Presentations & Student Deadline \\
\bottomrule
\caption{Project Timeline with Personal Work Deadlines}
\end{longtable}

\end{document}