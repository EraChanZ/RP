\section{Your contribution}
Typically in computer science the third section contains an exposition of the main ideas: the development of a theory, the analysis of the problem with some proofs, a new algorithm, and potentially a theoretical analysis of its properties.

Some more detailed suggestions for typical types of contributions in computer science are described below.

\subsection*{Experimental work}
Such section will mostly contain a description of the methods or algorithms that you will be comparing. Not all methods need to be described in detail (if appropriate references are available). Nonetheless, make sure to reveal sufficient details for a reader who is not familiar with these methods so that they can (1) get a general understanding of the methods and differences between them, and (2) understand your explanation of your results and conclusions.

\subsection*{Improvement of an idea}
Such section would require you to explain in detail how your improvement works. If your idea is based on some observation that can be proven, this is a good place to provide that proof (for instance of the correctness of your approach).

\subsection*{Literature survey}
If your contribution is a literature survey, then the organization of these ``middle'' sections very much depends on the way you want to present and organize the discussed literature. First, try to cluster papers that are similar in some aspect. Then, think how these clusters are related. Finally, think of a good order to discuss these clusters. This is sometimes called a bottom-up approach to writing a paper.

In addition, you may try to think about the organization of the literature from a top-down perspective: try to ``take a step back'' and think about the field and its important questions, and construct a hierarchical categorization of this field.

You should make your specific contribution clear: is it a new organization of the literature? identification of open problems or challenges? new parallels or generalizations? an analysis of pros and cons of different methods? 

Do not forget to give this section another name, for example relating to the method or the idea that you are presenting.