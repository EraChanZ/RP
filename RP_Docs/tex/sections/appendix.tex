\appendix
\section{Some further guidelines for your paper}

\begin{itemize}
\item Make sure to read the manual for CSE3000 Research Project at least once (note for example the instructions on the maximum length -- less can be more!).
\end{itemize}

\subsection{Reference use}
\begin{itemize}
\item use a system for automatically generating the bibliographic information from your database (a reference manager): BibTex, Zotero, EndNote, Papers, \ldots;
\item all ideas, fragments, figures, and data quoted from other work must be appropriately referenced;
\item literal quotations must be placed inside quotation marks and include the exact page numbers;
\item paraphrases cannot be too close to the original wording;
\item every reference in the text (such as this one \cite{example}) corresponds to an item in the bibliography and vice versa.
\end{itemize}

\subsection{Structure}
\begin{itemize}
\item paragraphs are well-constructed;
\item each paragraph discusses one topic;
\item they start with clear topic sentences;
\item they are organized into a clear structure;
\item there is a clear line of argumentation following from research question to the conclusions;
\item existing scientific literature is reviewed critically.
\end{itemize}

\subsection{Style}
\begin{itemize}
\item you should use proper English: your paper should be grammatically correct, without spelling errors and lexical mistakes (make sure to run a grammar and spell checker before submission);
\item you should write objectively;
\item you should attempt to write in an engaging manner: for instance, make sure to vary the length of the sentences, mix active and passive voice;
\item your sentences should not be unnecessarily complicated (e.g. not too long), they should refrain from ambiguity.
\end{itemize}

\subsection{Tables and figures}
\begin{itemize}
\item they should have a number and a caption;
\item they should be referred to at least once in the text;
\item if copied, they must contain a reference to the source;
\item they should be interpretable on their own (by means of labeled axes, descriptive legend, etc.);
\end{itemize}

\section{Gathering relevant literature}
A rule of thumb for dealing with the literature: scan about 10--20 papers: read the titles, abstracts, parts of introduction and conclusions, and categorize their contributions. Some of these should be studied more closely: read around 5 conference papers or equivalent completely (you should be able to summarize their contribution in your own words); study around 2 of them in-depth (you should be able to explain them in detail and criticize contributions). This process may result in 5--20 references, possibly even more if the project is a literature study. In case you need to gather a large number of references we suggest that you read online about two useful techniques: \textit{pearl growing} and \textit{snowballing}.
